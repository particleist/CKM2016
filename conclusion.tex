\section{Conclusion}
The last two decades have seen enormous progress in the understanding of $B$ meson mixing and mixing-related
$CP$ violation, both in terms of precise experimental measurements of the underlying constants of nature and in terms of 
the theoretical understanding of their values within the Standard Model. We now have precise measurements 
or stringent limits on the mass splitting, width splitting, and mixing phase in both the $B^0$ and $B^0_s$ systems,
while mixing-induced $CP$ violation is being precisely measured or constrained in an ever increasing number of final states.
With the LHCb upgrade~\cite{Bediaga:1443882} and Belle~II~\cite{Abe:2010gxa} detectors due to come online in the next few years,
and none of the fundamental measurements
yet systematically limited, we can expect this progress to continue. Important contributions, particularly as regards
$\phi_s$ and $\Delta\Gamma_d$ can also be expected from CMS and ATLAS, and in particular it is realistic to expect the $B^0_s$ mixing
phase to be measured significantly away from zero even at the Standard Model value within the next decade. Recently
the LHCb collaboration has proposed a Phase~II upgrade of its detector~\cite{Aaij:2244311}, to take data in the HL-LHC period,
which would collect 300~fb$^{-1}$, and enable not only a single-experiment observation of $\phi_s$ in multiple independent
decay modes, but also make it possible to see evidence for a non-zero $\Delta\Gamma_d$ at its Standard Model value.

Computation and further control of subleading corrections in the extraction of CKM angles and matrix elements from $B$ mixing has proceeded apace with new state-of-the-art lattice calculations. The former, particularly extraction of $\sin 2\beta$ from $b \to c\bar{c} s$ decays, has been the subject of various recent studies, with the emerging conclusion that theory uncertainties are likely smaller than the attainable experimental precision. Recent direct calculations as well as new flavor symmetry analyses will provide complementary handles to decide this question. State-of-the-art lattice results for $B$-mixing matrix elements indicate a possible evolving tension between predicted and measured mass splittings $\Delta m_{d,s}$, or in the context of the $b \to d$ CKM unitarity triangle, a tension between $\Delta m_{s,d}$ and $\epsilon_K$ allowed regions. Apart from constrained minimal violation theories, which seem unable to account for this tension, at present potential theoretical implications are relatively unexplored. 

%Particle physics is now roughly where the French monarchy was in 1788, and CERN is Versailles. But who will be our Robespierre?
