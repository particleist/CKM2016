\section{Introduction}
Neutral beauty mesons may oscillate into their antiparticles, so that the 
physical states (those with well-defined masses and lifetimes) are admixtures of the flavor eigenstates. This mixing is parametrized by the magnitudes 
of the dispersive and absorptive components of the $\langle B | H |\bar{B}\rangle$ transition amplitude -- a box loop in the standard model (SM) -- as well as their relative phase, denoted $\phi_{d,s}$ in the $B^0$ and $B^0_s$ systems, respectively. The dispersive component generates mass splittings of the physical states, $\Delta m_{(d,s)}$, and is sensitive to heavy off-shell contributions from new physics (NP). The absorptive component generates width splittings $\Delta\Gamma_{(d,s)}$. It arises only from on-shell internal charm and up quark contributions, and is therefore less sensitive to possible NP effects.

%Neutral beauty mesons may oscillate into their antiparticles via loop level processes, so that the 
%physical states (those with well defined masses and lifetimes) are admixtures of the flavor eigenstates. This gives
%rise to both mass ($\Delta m_{(d,s)}$) and width ($\Delta\Gamma_{(d,s)}$) splittings in the $B^0_d$ and $B^0_s$ systems, as well
%as two types of related $CP$ violation. 

Two types of mixing-related $CP$ violation arise in these systems. The first, $CP$ violation in mixing, occurs when the meson and anti-meson have different
probabilities to oscillate into each other, and is predicted to be very close to zero in the SM with a very
small theoretical uncertainty. The second, $CP$ violation in the interference of decay and mixing, occurs when both the meson
and anti-meson can decay to the same final state, and the decay paths with and without intermediate mixing interfere. This kind of $CP$
violation is highly sensitive to the phase, $\phi_{d,s}$, and its absolute size depends on the final state in question.

Aside from their intrinsically fundamental nature, measurements of mass and width splittings and mixing-related $CP$ violation are
of great interest because many of the observables can be predicted very precisely in the SM, and because new particles or force-carriers
beyond the SM (BSM) can alter these predictions in experimentally observable ways. Many different experimental collaborations have contributed
to our understanding of mixing in the $B$ system, from the initial discovery of $B$ mixing by the ARGUS collaboration~\cite{Prentice:1987ap}, to precise measurements
of $\Delta m_{(d,s)}$ and the CKM-angles $\alpha$ and $\beta$ at the $B$-factories and Tevatron, to recent precise measurements
of $\phi_s$, $\Delta\Gamma_s$, and first precise mixing-related measurements of the CKM-angle $\gamma$ at ATLAS, CMS, and LHCb. At the same
time, great progress has been made in making more precise theoretical predictions of the SM values of many of these quantities, making
these experimental measurements sensitive probes of BSM physics.

The remainder of this document covers the current status of experimental measurements for each observable of
interest, as well as their near-term outlook. Section~\ref{sec:phisdgs} covers measurements of $\phi_s$ and $\Delta\Gamma_s$, section~\ref{sec:dmdgd} 
measurements of $\Delta m_{(s,d)}$. Section~\ref{sec:photpol} covers the measurements of photon polarization in radiative decays, while
sections~\ref{sec:alpha}~to~\ref{sec:gamma} cover measurements of the CKM angles $\alpha$, $\beta$, and $\gamma$, respectively.
For historical reasons, measurements of $CP$ violation in neutral meson mixing are covered in the proceedings of WG2~\cite{WG2PROC}. We then discuss
ongoing theoretical developments relevant to our WG in Sec.~\ref{sec:theory}. Finally we conclude,
and discuss the medium to long-term outlook for mixing-related measurements in the $B$ system.


