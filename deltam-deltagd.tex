\section{Measurements of $\Delta m_{(d,s)}$ and $\Delta\Gamma_d$}
\label{sec:dmdgd}

The neutral $B$ meson mass splittings $\Delta m_{(d,s)}$ have been precisely measured by BaBar, Belle, CDF, D0, and LHCb.
The world average~\cite{HFAG} of these measurements is currently dominated by the LHCb determinations,
based on an analysis~\cite{} of semileptonic $B^0$ decays in the case of $\Delta m_d$, and based on an analysis~\cite{} of $B^0_s \to \D^-_s \pi^+$ decays
in the case of $\Delta m_s$. LHCb finds
\begin{equation}
\Delta m_d = (0.505 \pm 0.0021 \pm 0.0010) \textrm{ps}^{-1}\,,\phantom{space} 
\Delta m_s = (17.768 \pm 0.023 \pm 0.006) \textrm{ps}^{-1}
\end{equation}
where the first uncertainty is statistical, and the second systematic. Neither analysis is systematics limited,
and LHCb is expected to update both measurements in the future. Belle-II will also be able to make a significant contribution
to a precise measurement of $\Delta m_d$. A measurement of $\Delta m_s$ will be difficult for ATLAS and CMS because of a lack of
efficient triggers for purely hadronic decays, but may become possible in the HL-LHC era once their first-level tracking triggers come online,
which would provide an important independent cross-check of LHCb's measurement.

The $B^0$ width splitting $\Delta\Gamma_d$ is predicted~\cite{} to be $\Delta\Gamma_d/\Gamma_d = (4 \pm 1)\cdot 10^{-3}$ in the SM, and measuring it
is therefore a useful null test of the SM~\cite{}. Such tests are particularly important as a non-null value of $\Delta\Gamma_d/\Gamma_d$ could have important
implications for the interpretation of $CP$ violation in the mixing of $B^0$ mesons, particularly in the context~\cite{} of the D0 dimuon asymmetry measurement.
LHCb has analyzed the effective decay-times of $B^0 \to J/\psi K^{*0}$
and $B^0 \to J/\psi K^0_\textrm{S}$ decays and obtains~\cite{} $\Delta\Gamma_d/\Gamma_d = (-4.4 \pm 2.5 \pm 1.1)\cdot 10^{-2}$,
while the current WA is dominated by the ATLAS measurement~\cite{}
of $\Delta\Gamma_d/\Gamma_d = (-0.1 \pm 1.1 \pm 0.9)\cdot 10^{-2}$, where the first uncertainties are statistical and the second systematic.
While the systematic uncertainties are dominated by simulation sample sizes, both measurements are still an order
of magnitude away from probing the SM prediction, so it will be important that even the subdominant
systematics scale with luminosity if we hope to one day observe a non-null $\Delta\Gamma_d/\Gamma_d$ at its SM value. 
 
The BaBar Collaboration studies the $B_d^0-\overline{B_d}^0$ oscillations to
test the conservation of the $CPT$ symmetry\cite{babar_cpt}. At the lowest order in $|q/p|-1$
and $z$, the two mass eigenstates can be written:
\begin{eqnarray}
B_H & = & (p\sqrt{1+z} \; B^0 - q\sqrt{1-z} \; \overline{B}^0) / \sqrt{2} \\
B_L & = & (p\sqrt{1-z} \; B^0 + q\sqrt{1+z} \; \overline{B}^0) / \sqrt{2} 
\end{eqnarray}
where:
\begin{equation}
  |q/p| = 1 - \frac{2 \Im(m_{12}^* \Gamma_{12})}{4|m_{12}|^2 + |\Gamma_{12}|^2}, \hspace{1cm}
  z = \frac{(m_{11}-m_{22}) - i (\Gamma_{11} - \Gamma_{22})/2}{\Delta m - i \Delta \Gamma /2}.
\end{equation}

The test is performed by fitting the $C$ and $S$ parameters of the $CP$ violation in the
interference between mixing and decay in $B \to c\bar{c} K^0_{S,L}$, when the other $B$
in the event decays to the $\ell^{\pm}X$ final state, separating the cases when the decay
of the $B$ to the $CP$-eigenstate happens before or after the decay of the other $B$ to
the flavor-specific final state. The assumption $|\overline{A}/A|=1$, where $A$ ($\overline{A}$)
is the amplitude of $B^0 \to c\bar{c} K^0$ ($\overline{B}^0 \to c\bar{c} \overline{K}^0$) is
not enforced.

The final result is:
\begin{eqnarray}
  \Im (z) & = &\phantom{-} 0.010 \pm 0.030 (\mbox{stat}) \pm 0.013 (\mbox{syst}) \, ,\\
  \Re (z) & = & -0.065 \pm 0.028 (\mbox{stat}) \pm 0.014 (\mbox{syst}) \, ,\\
  |\overline{A}/{A}| & = & \phantom{-} 0.999 \pm 0.023 (\mbox{stat}) \pm 0.017 (\mbox{syst}) \, ,
\end{eqnarray}
in good agreement with $CPT$ conservation.
  
