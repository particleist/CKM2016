\section{Measurements of the CKM angle $\gamma$}
\label{sec:gamma}
Measurements of the CKM angle $\gamma$ require the interference of $b\to u$ and $b\to c$ transitions.
Such interference can occur in the decays of charged as well as neutral $B$ hadrons, in tree-dominated
transitions as well as transitions where both tree and loop diagrams contribute.
The measurement of $\gamma$ from the tree-dominated decays of $B^\pm$ mesons, covered in the 
proceedings of WG5~\cite{WG5PROC}, has a particular importance as it allows a purely tree-level\footnote{Higher order box corrections~\cite{ZupanBrodGamma}
only enter at $\delta\gamma/\gamma \approx 10^{-7}$, well beyond any current or future experimental sensitivity.}
determination of the apex of the unitarity triangle, and therefore a test of the self-consistency of the CKM
mechanism of $CP$ violation when compared with determinations of $\alpha$ and $\beta$ in transitions where
both tree and loop diagrams contribute. It is also, however, possible to measure $\gamma$ by exploiting
$CP$ violation in the interference of mixing and decay of neutral $B$ mesons. These time-dependent determinations
of $\gamma$ are particularly powerful in the case of $B^0_s$ mesons, because the large width difference between
the light and heavy $B^0_s$ eigenstates makes additional $CP$ observables available compared
to the $B^0$ system and allows for a determination of $\gamma$ with fewer ambiguities. 
%While these are not
%strictly tree-level determinations, as the mixing of neutral $B$ mesons

A particularly powerful time-dependent measurement of $\gamma$, described in detail in these proceedings~\cite{DSKPROC}, utilizes the decay $B^0_s \to D^\pm_s K^\mp$.
In this case the $b\to u$ and $b\to c$ transitions are both of order $\lambda^3$, leading to large interference, and
the large value of $\Delta\Gamma_s$ allows for a determination of $\gamma$ with only a twofold ambiguity. 
The preliminary result obtained with the full Run~I LHCb dataset, which shows $3.6\sigma$ evidence for $CP$-violation in this mode, is
\begin{equation}
\gamma      = (127_{-22}^{+17})^\circ\,,\phantom{space}
\strong = (  358_{-16}^{+15})^\circ\,,\phantom{space}
\rdsk   = 0.37_{-0.09}^{+0.10}\,,\phantom{space}
(68.3\% \textrm{CL})\phantom{space}  
\end{equation}
\begin{equation}
\gamma      = (127_{-50}^{+33})^\circ\,,\phantom{space}
\strong = (  358_{-33}^{+31})^\circ\,,\phantom{space}
\rdsk   = 0.37_{-0.19}^{+0.19}\,,\phantom{space}
(95.4\% \textrm{CL})\phantom{space}  
\end{equation}
where $\strong$ is the $CP$-conserving angle between the $b\to u$ and $b\to c$ transitions,
$\rdsk$ is the amplitude ratio of the interfering diagrams, and the intervals for the angles are expressed modulo $180^\circ$.
The uncertainties are a combination of statistical and systematic ones; the statistical uncertainties dominate
and all systematic uncertainties are expected to scale with luminosity for the foreseeable future.

While not the most sensitive single-mode determination of $\gamma$, $B^0_s \to D^\pm_s K^\mp$ plays a similar
role in the overall LHCb combination~\cite{LHCb-PAPER-2016-032} of $\gamma$ to that of the GGSZ measurement. Because of their twofold ambiguity, these measurements
select the ``correct'' solution among the ones allowed by the most precise ADS/GLW measurement~\cite{LHCb-PAPER-2016-003}. For this reason the determination
of $\gamma$ from $B^0_s \to D^\pm_s K^\mp$, which is only possible at LHCb, will remain a key measurement for both the current
and upgraded LHCb detectors. LHCb is also pursuing a measurement of time-dependent $CP$-violation in
the decay mode $B^0 \to D^\pm \pi^\mp$, described in these proceedings~\cite{BDPIPROC}, but no results are available yet.
This measurement, previously performed by BaBar~\cite{Aubert:2005yf}, \cite{Aubert:2006tw} and Belle~\cite{Bahinipati:2011yq}, \cite{Ronga:2006hv}
is much less sensitive than $B^0_s \to D^\pm_s K^\mp$, both because of smaller interference and because
the small value of $\Delta\Gamma_d$ leads to fewer accessible $CP$-observables. The much smaller $CP$ asymmetry
also makes this measurement particularly sensitive the asymmetries in the flavor tagging of $B^0$ and $\bar{B^0}$ mesons.
Nevertheless, it is expected that both LHCb and Belle-II will carry out this measurement in the future.
