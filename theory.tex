%% Please use the skeleton file you have received in the
%% invitation-to-submit email, where your data are already
%% filled in. Otherwise please make sure you insert your
%% data according to the instructions in PoSauthmanual.pdf
%\documentclass{PoS}
%
%\usepackage{amssymb, amsmath}
%
%\title{Summary of Working Group 4 : mixing and mixing-related $CP$ violation in the $B$ system}
%
%\ShortTitle{All SM everything}
%
%\author{\speaker{Alessandro Gaz}\\
%        Nagoya\\
%        E-mail: \email{gaz@hepl.phys.nagoya-u.ac.jp}}
%\author{
%        \speaker{Vladimir V. Gligorov}\thanks{I would like to thank my mum, dad, and trade union representative for their support. Couldn't have done it without you!}\\
%        LPNHE, Universit\'{e} Pierre et Marie Curie, Universit\'{e} Paris Diderot, CNRS/IN2P3, Paris, France\\
%        E-mail: \email{vgligoro@lpnhe.in2p3.fr}
%        }
%\author{
%        \speaker{Dean Robinson}\thanks{Show me an ambulance dashing off into the night, and I shall show you the theorists bravely in pursuit.}\\
%        University of Cincinnati, Cincinnati, Ohio, USA\\
%        E-mail: \email{dean.robinson@uc.edu}\\
%        }
%
%%\author{Another Author\\
%%        Affiliation\\
%%        E-mail: \email{...}}
%
%\abstract{$B$ mesons mix and in mixing do wonderful and amazing things.}
%
%\FullConference{9th International Workshop on the CKM Unitarity Triangle\\
%		28  November - 3 December 2016\\
%		Tata Institute for Fundamental Research (TIFR), Mumbai, India}
%
%
%\begin{document}
\section{Theory Developments}
Computation and further control of subleading corrections in the extraction of CKM angles and matrix elements  has proceeded apace with new state-of-the-art lattice calculations 
%for $\Delta M_{d,s}$, and therefrom $V_{ts}$ and $V_{td}$.

\subsection{Lattice Calculations}

The most precise computation to date for $|V_{td}|$ and $|V_{ts}|$, or alternatively $\Delta M_{d,s}$, is recently available from the Fermilab/MILC collaboration. These predictions are achieved from direct calculation of up to five neutral $B$ mixing hadronic matrix elements, $\langle \bar{B} | \mathcal{O}^q_i | B\rangle$, with $\mathcal{O}^q_i$ combinations of scalar, pseudoscalar, vector and axial vector four-quark operators as required by the theory of interest. In the SM, the oscillation frequency
\begin{equation}
	\label{eqn:SMDM}
	\Delta M_q = \frac{G_F^2 m_W^2 M_{B_q}}{6 \pi^2} S_0(m_t^2/m_W^2) \eta_{2 B}|V_{tb}V^*_{tq}| f_{B_q}^2 \hat{B}^{(1)}_{B_q}\,,
\end{equation}
in which $\hat{B}^{(1)}_{B_q}$ is a renormalization-improved bag parameter associated with the left-handed quark current produced by $\mathcal{O}^q_1$, while $S_0(x_t)$ and $\eta_{2B}$ encode known electroweak and perturbative QCD corrections, respectively. Computation of $f_{B_q}\sqrt{\hat{B}^{(1)}_{B_q}}$ may be combined with either direct measurements of $\Delta M_{s,d}$ or CKM global fits to test the self-consistency of data with lattice calculations. One may also extract $|V_{td}/V_{ts}|$ or $\Delta M_d/\Delta M_s$ via the flavor $SU(3)$ breaking ratio 
\begin{equation}
\xi = \sqrt{f_{B_s}^2 \hat{B}^{(1)}_{B_s}/ f_{B_d}^2 \hat{B}^{(1)}_{B_d}}\,,
\end{equation}
in which many theory uncertainties cancel.

This lattice calculation is performed for $N_f = 2+1$ light quark flavors in `asqtad' ensembles. Non-perturbative renormalization effects are included, while two-loop chiral-continuum extrapolation of the lattice results determines the physical limits. Presentation of all crucial details may be found in Ref.~[]. Present results are
\begin{gather}
	f_{B_d}\sqrt{\hat{B}^{(1)}_{B_d}} = 227.7(9.5)(2.3)~\text{MeV} \qquad f_{B_s}\sqrt{\hat{B}^{(1)}_{B_s}} = 274.6(8.4)(2.7)~\text{MeV}\,,\nonumber\\
		\xi = 1.206(18)(6)\,,
\end{gather}
currently the most precise predictions to date. Combination of these results with $|V_{tq}|$ results from CKM global fits yields 
\begin{gather}
	\Delta M^{\text{CKM}}_d = 0.630(53)(42)(5)(13)~\text{ps}^{-1}\,, \qquad \Delta M^{\text{CKM}}_s = 19.6(1.2)(1.0)(0.2)(0.4)~\text{ps}^{-1}\,,\nonumber \\
	\Delta M^{\text{CKM}}_d/\Delta M^{\text{CKM}}_s = 0.0321(10)(15)(0)(3)\,,
\end{gather}
in an approximately $2\sigma$ tension with direct measurements of these parameters, viz.
\begin{equation}
	\Delta M^{\text{HFAG}}_d = 0.5064(19)~\text{ps}^{-1}\,,\qquad \Delta M^{\text{HFAG}}_s = 17.757(21)~\text{ps}^{-1}\,.
\end{equation}
Alternatively, using the direct measurements for $\Delta M_{d,s}$ produces CKM predictions
\begin{gather}
	|V_{td}| = 8.00(33)(2)(3)(8)\times 10^{-3}\,,\qquad |V_{ts}| = 39.0(1.2)(0.0)(0.2)(0.4)\times 10^{-3}\,,\nonumber\\
	|V_{td}/V_{ts}| = 0.2052(31)(4)(0)(10)\,,
\end{gather}
approximately $\sim2\sigma$ below the results from global CKM fits. In particular, in the context of the $b \to d$ unitarity triangle, this tension manifests as a tension between the allowed regions for the CP violating parameter $\epsilon_K$ and $\Delta M_{d}/\Delta M_s$, with potentially interesting theory implications. 

These same lattice calculations also generate predictions for $B_{d,s} \to \mu\mu$ and $\Delta\Gamma_{d,s}$, with some mild tensions for $B_d$ decays. Further improved calculations for $N_f = 2 + 1 + 1$ that include the charm quark sea effects and physical quark masses are anticipated.

\subsection{Constrained MFV theories}
Given the possible tensions in the unitarity triangle between $\epsilon_K$ and $\Delta M_{d,s}$, it is informative to consider which classes of BSM theories could account for such tension. A class involving near-minimal BSM contributions are models of constrained minimal flavor violation (CMFV). In these theories, the SM yukawas $Y_u$ and $Y_d$ are treated as $U(3)\times U(3)$ flavor violating spurions, generating the sole source of CP violation.  BSM effects are encoded in higher dimensional SM effective operators.

This class of theories preserves the unitarity structure of the CKM matrix. The precisely measured CKM matrix elements for the first two generations imply the relation, at percent level precision,
\begin{equation}
	R_t \equiv \bigg|\frac{V_{tb}^*V_{td}}{V_{cb}^*V_{cd}}\bigg| \simeq \frac{|V_{td}/V_{ts}|}{\lambda}\,,
\end{equation}
where $\lambda$ is the usual Wolfenstein parameter. The ratio $|V_{td}/V_{ts}|$ is determined precisely via Eq.~\eqref{eqn:SMDM} from direct measurements of $\Delta M_d/ \Delta M_s$ and lattice computations of $\xi$. Thus, measurements of $\Delta M_{d,s}$ and the time-dependent $CPV$ observable $S_{\psi K_S}$, which determines $\sin(2\beta)$ (see below), fully determine a `universal unitarity triangle' (UUT) for the $b \to d$ system for all CMFV theories. 

The electroweak loop function $S_0(x_t)$ in $\Delta M_{d,s}$~\eqref{eqn:SMDM} also appears in the CP violating parameter $\epsilon_K$. CMFV replaces $S_0(x_t)$ with a generalized universal function $S(v)$, bounded below by $S_0(x_t)$ in most compelling BSM scenarios. The UUT is, however, independent of the electroweak loop function, as it drops out of the $\Delta M_{d}/\Delta M_s$ ratio. Consequently, in CMFV theories, there is an extra degree of freedom between the predictions for $|V_{ub}/V_{cb}|$ and the CKM angle $\gamma$, both fixed by the UUT, and the constraints from measurements of $\epsilon_K$. 

This extra freedom is, however, found to be insufficient to relax the tension between $\Delta M_{d,s}$ and $\epsilon_K$ generated in the FNAL/MILC lattice results, when applied togather with the bound $S(v) \ge S_0(x_t)$. One may take either $\Delta M_{s,d}$ or $\epsilon_K$ direct measurements as inputs, and thereby determine all other CKM matrix elements as functions of $S(v)$ via the UUT constraints: $\Delta M_{d,s}$ direct measurements imply an upper bound on $\epsilon_K$ that is too small compared to data; $\epsilon_K$ data implies lower bounds on $\Delta M_{d,s}$, that are above current measurements. If the tension between lattice and experimental data persists in the unitarity triangle, it will become imperative to consider new sources of flavor violation in $\Delta F = 2$ processes, beyond CMFV models.

\subsection{Precision control of Penguin Pollution}
%Extraction of the mixing phases $\phi_{d,s}$ can be achieved with standard techniques that make use of observables sensitive to CP violation in mixing and decay. In particular, 
The time-dependent $CPV$ observable $S_{\psi K_S}$, generated by the interference of $B$-mixing and $B_d \to J/\psi K_S$ decay amplitudes, has long been considered a golden mode for extraction of $\phi_d$, via the relation $S_{\psi K_S} = \sin (\phi_d + \delta \phi_d)$. The latter `penguin pollution' phase is expected to be CKM and loop-suppressed, yielding a clean measurement of $\phi_d \simeq 2 \beta$ in the SM. Similar techniques may be used to extract $\phi_s$ from, e.g., $B_s \to J/\psi \phi$. Recent estimates place $\delta \phi_d \lesssim 1^\circ$, which is near to or larger than the expected precision of upcoming measurements, requiring theoretical control of these terms.

Two different paths have been pursued to control these penguin pollution terms. A direct calculational approach makes use of an OPE-type approach. This analysis integrates out the $u$-quark loop, associated with this pengiun pollution term, on the basis that the typical momentum flow is large. One obtains a factorization formula for the penguin contributions, and one may show that soft and collinear divergences formally cancel or factorize up to $\Lambda_{\text{qcd}}/m_{J/\psi}$ corrections. Large-$N_c$ arguments permit estmation of remaining uncertainties, yielding $\delta \phi_d < 0.68^\circ$ anf $\delta \phi_s \lesssim 1^\circ$. 

A second, complementary path makes use of light quark flavor symmetries to constrain or eliminate penguin pollution effects in particular observables. For example on may make use of $U$-spin symmetry to  control pollution of $S_{\psi K_S}$ with $B \to J/\psi \pi$ data, up to symmetry breaking effects. Full $SU(3)$ analyses that treat all flavor symmetry breaking effects, or subsets of them, can be used to control penguin pollution. A simultaneous fit predicts $\delta \phi_d \lesssim 0.6^\circ$. Other flavor symmetry results permit extraction of $\phi_d$ with  penguin pollution suppressed by $SU(3)$ breaking, but require precision measurement of $CP$-averaged rates in the charged and neutral $B \to J/\psi K$ and $B \to J/\psi \pi$ systems.

\subsection{Lessons of isospin violation}
Measurement of CP averaged rates at the percent-level precision required for the use of above-mentioned flavor $SU(3)$ results in turn requires careful inclusion of isospin violating effects in the ratio of neutral and charged $B$ production. Knowledge of this ratio, $r_{+0} = f_{+-}/f_{00}$, is crucial for precision measurement of branching ratios:  charged (neutral) $B$ branching ratios are sensitive to $1 + r_{+0}$ ($1 + 1/r_{+0}$),  expected to be of the order of a few percent. Control of this ratio can be achieved via e.g. counting single versus double tagged semileptonic $B$ decays or measuring relative branching ratios for inclusive semileptonic processes, in which isospin violation is expected to be suppressed. 

%\end{document}

\endinput
%The $\Delta F = 2$  mixing operators encode NP effects in terms of a single flavor universal function, $S(v)$ that replaces 

%Only Y_u, Y_d FV spurions plus SM EFT operators.
% Delta M_s/Delta M_d plus S_{\psi K_S} determine universal $b \to d$ unitarity triangle
%

% <need to check> , and in particular  As such, the ratio , is insensitive to this NP. The side length $R_t \equiv |V_{tb} V^*_{tq}/V_{cb} V^*_{cq}|$ and $\sin2\beta$-- determined by $\Delta M_{d,s}$ and $S_{\psi K_S}$, respectively -- can, however, be modified. 



