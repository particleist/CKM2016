%% Please use the skeleton file you have received in the
%% invitation-to-submit email, where your data are already
%% filled in. Otherwise please make sure you insert your
%% data according to the instructions in PoSauthmanual.pdf
%\documentclass{PoS}
%
%\usepackage{amssymb, amsmath}
%
%\title{Summary of Working Group 4 : mixing and mixing-related $CP$ violation in the $B$ system}
%
%\ShortTitle{All SM everything}
%
%\author{\speaker{Alessandro Gaz}\\
%        Nagoya\\
%        E-mail: \email{gaz@hepl.phys.nagoya-u.ac.jp}}
%\author{
%        \speaker{Vladimir V. Gligorov}\thanks{I would like to thank my mum, dad, and trade union representative for their support. Couldn't have done it without you!}\\
%        LPNHE, Universit\'{e} Pierre et Marie Curie, Universit\'{e} Paris Diderot, CNRS/IN2P3, Paris, France\\
%        E-mail: \email{vgligoro@lpnhe.in2p3.fr}
%        }
%\author{
%        \speaker{Dean Robinson}\thanks{Show me an ambulance dashing off into the night, and I shall show you the theorists bravely in pursuit.}\\
%        University of Cincinnati, Cincinnati, Ohio, USA\\
%        E-mail: \email{dean.robinson@uc.edu}\\
%        }
%
%%\author{Another Author\\
%%        Affiliation\\
%%        E-mail: \email{...}}
%
%\abstract{$B$ mesons mix and in mixing do wonderful and amazing things.}
%
%\FullConference{9th International Workshop on the CKM Unitarity Triangle\\
%		28  November - 3 December 2016\\
%		Tata Institute for Fundamental Research (TIFR), Mumbai, India}
%
%
%\begin{document}
\section{Theory Developments}
Computation and further control of subleading corrections in the extraction of CKM angles and matrix elements  has proceeded apace with new state-of-the-art lattice calculations 
%for $\Delta M_{d,s}$, and therefrom $V_{ts}$ and $V_{td}$.

\subsection{Lattice Calculations}
<Insert more history>

The most precise computation to date for $V_{td}$ and $V_{ts}$ is recently available from the Fermilab/MILC collaboration [cite]. Extraction of these CKM matirx elements is achieved from direct calculation of five neutral $B$ mixing hadronic matrix elements, that depend on $\Delta M_{d,s}$ and the mixing phases $\phi_{q} = \text{arg}[V_{tb}V^*_{tq}/V^*_{tb}V_{tq}] $. This calculation is performed for $N_f = 2+1$ light quark flavors, in asqtad ensembles. Non-perturbative renormalization effects are included; two-loop chiral-continuum extrapolation of the lattice results determines the physical limits.

At present these new results indicate a mild $\sim 2 \sigma$ tension between $\Delta M_{d,s}$ results and the direct measurements of the CP violating parameter $\epsilon_K$, as shown  in the $b \to d$ unitarity triangle in Fig. [], with potentially interesting theory implications. These lattice calculations also generate predictions for $B_{d,s} \to \mu\mu$ and $\Delta\Gamma_{d,s}$, with some mild tensions for $B_d$ decays [clarify]. Further improved calculations for $N_f = 2 + 1 + 1$ that include the charm quark sea effects and physical quark masses are anticipated.

\subsection{Constrained MFV theories}
Given the possible tensions in the unitarity triangle between $\epsilon_K$ and $\Delta M_{d,s}$, it is informative to consider which classes of BSM theories could account for such tension. A class involving near-minimal BSM operators are constrained minimal flavor violation (CMFV) models. In these theories, the SM yukawas $Y_u$ and $Y_d$ are treated as $U(3)\times U(3)$ flavor violating spurions, generating the sole source of CP violation. <need to check>  BSM effects are encoded in higher dimensional SM effective operators, and in particular $\Delta F = 2$ mixing operators encode NP effects in terms of a single flavor universal function. As such, the ratio $V_{ub}/V_{cb}$, and therefore the CKM angle $\gamma$, is insensitive to this NP. The side length $R_t \equiv |V_{tb} V^*_{tq}/V_{cb} V^*_{cq}|$ and $\sin2\beta$-- determined by $\Delta M_{d,s}$ and $S_{\psi K_S}$, respectively -- can, however, be modified. 

This extra freedom in the unitarity triangle is, however, found to be insufficient to relax the tension between $\Delta M_{d,s}$ and $\epsilon_K$: Fixing a large $V_{cb}$ from $\epsilon_K$ data pushes $\Delta M_{d,s}$ above current lattice expectations; Fixing a small $V_{cb}$ from $\Delta M_{d,s}$ pushes $\epsilon_K$ to be too small. If the tension between lattice and experimental data persists in the unitarity triangle, it will become imperative to consider new sources of flavor violation in $\Delta F = 2$ processes, beyond CMFV models.

\subsection{Precision control of Penguin Pollution}
%Extraction of the mixing phases $\phi_{d,s}$ can be achieved with standard techniques that make use of observables sensitive to CP violation in mixing and decay. In particular, 
The time-dependent $CPV$ observable $S_{\psi K_S}$, generated by the interference of $B$-mixing and $B_d \to J/\psi K_S$ decay amplitudes, has long been considered a golden mode for extraction of $\phi_d$, via the relation $S_{\psi K_S} = \sin (\phi_d + \delta \phi_d)$. The latter `penguin pollution' phase is expected to be CKM and loop-suppressed, yielding a clean measurement of $\phi_d \simeq 2 \beta$ in the SM. Similar techniques may be used to extract $\phi_s$ from, e.g., $B_s \to J/\psi \phi$. Recent estimates place $\delta \phi_d \lesssim 1^\circ$, which is near to or larger than the expected precision of upcoming measurements, requiring theoretical control of these terms.

Two different paths have been pursued to control these penguin pollution terms. A direct calculational approach makes use of an OPE-type approach. This analysis integrates out the $u$-quark loop, associated with this pengiun pollution term, on the basis that the typical momentum flow is large. One obtains a factorization formula for the penguin contributions, and one may show that soft and collinear divergences formally cancel or factorize up to $\Lambda_{\text{qcd}}/m_{J/\psi}$ corrections. Large-$N_c$ arguments permit estmation of remaining uncertainties, yielding $\delta \phi_d < 0.68^\circ$ anf $\delta \phi_s \lesssim 1^\circ$. 

A second, complementary path makes use of light quark flavor symmetries to constrain or eliminate penguin pollution effects in particular observables. For example on may make use of $U$-spin symmetry to  control pollution of $S_{\psi K_S}$ with $B \to J/\psi \pi$ data, up to symmetry breaking effects. Full $SU(3)$ analyses that treat all flavor symmetry breaking effects, or subsets of them, can be used to control penguin pollution. A simultaneous fit predicts $\delta \phi_d \lesssim 0.6^\circ$. Other flavor symmetry results permit extraction of $\phi_d$ with  penguin pollution suppressed by $SU(3)$ breaking, but require precision measurement of $CP$-averaged rates in the charged and neutral $B \to J/\psi K$ and $B \to J/\psi \pi$ systems.

\subsection{Lessons of isospin violation}
Measurement of CP averaged rates at the percent-level precision required for the use of above-mentioned flavor $SU(3)$ results in turn requires careful inclusion of isospin violating effects in the ratio of neutral and charged $B$ production. Knowledge of this ratio, $r_{+0} = f_{+-}/f_{00}$, is crucial for precision measurement of branching ratios:  charged (neutral) $B$ branching ratios are sensitive to $1 + r_{+0}$ ($1 + 1/r_{+0}$),  expected to be of the order of a few percent. Control of this ratio can be achieved via e.g. counting single versus double tagged semileptonic $B$ decays or measuring relative branching ratios for inclusive semileptonic processes, in which isospin violation is expected to be suppressed. 

%\end{document}






